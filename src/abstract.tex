\documentclass[main.tex]{subfiles}

\begin{document}

% Abstract.
\begin{abstract}
  Concurrency is \emph{hard}. A concurrent system is a collection of programs that simultaneously
  operate on shared data. Concurrency introduces ambiguity in execution order, because these
  programs may be run across threads, processes, or, in the case of \textbf{distributed systems},
  networks. It is precisely this ambiguity that causes a certain class of failures known as race
  conditions. Race conditions manifest themselves in subtle ways in concurrent systems, but they can
  often have catastrophic consequences.

  Many programming languages provide fundamental abstractions such as locks, semaphores,
  and monitors to protect against race conditions. Some, like Rust~\cite{rust}, are even able to
  statically detect race conditions between concurrent threads. But none are able to
  \emph{guarantee} that race conditions will be exhaustively eliminated in a distributed system.
  For this reason, building correct distributed systems is extremely challenging.

  In this thesis, we will explore transactional programming languages and their application to
  building race-free distributed systems. In Chapter 1, we will discuss a novel transactional
  programming language called Caustic and its associated runtime. In Chapter 2, we will discuss its
  underlying distributed, transactional storage system, Beaker.
\end{abstract}

\end{document}