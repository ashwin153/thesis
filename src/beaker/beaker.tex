\documentclass[../main.tex]{subfiles}

\begin{document}
Beaker is a distributed, transactional database that serves as the underlying storage system for
Caustic. Beaker is fault-tolerant, configurable, and performant.

In Section 1, we motivate the design of Beaker through a discussion of consensus. In Section
2, we discuss the algorithm that Beaker uses to commit distributed transactions and provide a rough
proof of correctness and a description of its implementation. In Section 3, we evaluate the
performance of Beaker under a variety of workloads.

% Consensus.
\section{Consensus}
The only certainty in distributed systems is that machines will fail. The key challenge in building
fault-tolerant distributed systems is constructing reliable systems from unreliable components.
~\cite{reliability} Distributed databases replicate data across a cluster of machines. By storing
copies in different places, distributed databases are tolerant of individual machine failures.

Replication solves the problem of fault-tolerance, but creates another; the various replicas must be
kept consistent with each other. A na\"ive approach is to require all replicas to first agree to
apply any modifications made to the database. This would ensure that all replicas remained
identical. However, this approach is not fault-tolerant. If any replica were to fail, then no
modifications could ever be made to the database until it recovered. Instead, fault-tolerant
distributed databases require only that a \textbf{quorum}, or simple majority, of replicas to reach
agreement before modifications can be safely applied. This ensures that any quorum of replicas will
contain at least one that has the latest copy of the database while remaining tolerant of a minority
of individual machine failures. Reaching quorum agreement in a distributed system is known as
consensus, and it is a relatively well-studied problem. In this section, we explore different
consensus algorithms culminating with the approach taken in Beaker.

  \subsection{Terminology}
  We define a \textbf{proposal} as an abstract operation on a group of replicas and a
  \textbf{proposer} as the replica that initiates consensus on a proposal. Over the course of this
  discussion, we will gradually refine this definition of a proposal until we arrive at the one used
  in Beaker.

  Replicas communicate by sending messages. We assume that delivered messages cannot be reordered;
  if message $A$ was delivered before message $B$, then $A$ was sent before $B$. In practice, this
  assumption is satisfied by most networking protocols including TCP.

  \subsection{Two-Phase Commit}
  In Two-Phase Commit, the proposer \textbf{prepares} a proposal by first acquring locks on a quorum
  of replicas. If it successfully acquires all locks, the proposer informs all replicas to
  \textbf{learn} the proposal. When a proposal is learned, its operation is applied and all locks are
  subsequently released.

  Two-Phase Commit is not fault-tolerant. If the proposer were to fail after it successfully
  prepared a proposal but before it requested that it be learned, the locks that it acquired would
  never be released and no new proposals could ever be learned. We will see in Paxos how we can
  modify the protocol to guarantee fault-tolerance.

  \subsection{Paxos}
  Paxos makes two modifications to Two-Phase Commit to address its fault-intolerance. First, it
  associates each proposal with a monotonically-increasing, globally-unique \textbf{ballot} number.
  Proposals are totally-ordered and uniquely-identified by their \textbf{ballot}. Each replica
  keeps track of the latest ballot that it has seen. Second, it introduces an intermediate
  \textbf{accept} phase to the protocol. We will see that this additional phase allows the system to
  recover from proposer failure.~\cite{paxos}

  In Paxos, the proposer prepares a proposal by assigning it a ballot greater than any it has seen
  and sending its ballot to a quorum of replicas. If the ballot is greater than any it has seen, a
  replica \textbf{promises} not to accept any proposal less than it and returns any proposal that it
  has already accepted. Otherwise, the replica ignores the request. Intuitively, ballots function as
  a kind of implicit lock that the proposer holds until another proposer prepares a greater ballot.
  If a majority of replicas do not respond to its prepare request, the proposer retries with a
  greater ballot. Otherwise, the proposer selects a proposal to be accepted. If any replica
  returned an accepted proposal, then the proposer must select the latest accepted proposal and set
  its ballot to the one that it prepared. Otherwise, the proposer selects its own proposal.
  Intuitively, this allows the system to recover when a proposer fails after convinced a
  majority to accept its proposal but before it could be learned. A replica accepts a proposal if
  and only if it has not promised not to. When a replica accepts a proposal, it requests that all
  replicas learn it. A replica learns a proposal when a majority of replicas have requested that it
  be learned. When a replica is learned, its operation is applied and any accepted proposals are
  removed. Intuitively, this releases the implicit lock held by the proposer and consensus to begin
  on another proposal.

  Paxos guarantees that all non-faulty replicas will learn proposals in the same order. Often, this
  guarantee is unnecessary because a large number of operations on a distributed system are
  commutative, so they may be performed in any order. For example, reads and writes to different
  keys in a database may be performed in any order without compromising consistency. We will see in
  Generalized Paxos that we can exploit commutativity to improve performance.

  It is known that no deterministic fault-tolerant consensus protocol can guarantee progress in an
  asynchronous network.~\cite{consensus} Paxos is no exception. If a higher ballot is continuously
  prepared before any proposal can be accepted, no proposal will ever be learned. Implementations of
  Paxos typically elect a distinguished replica, called a \textbf{leader}, to which all other
  replicas forward their proposals to guarantee liveness. Whenever leaders fail, replicas run an
  instance of Paxos to acquire leadership of the cluster. The reliance on the existence of a single,
  stable leader is both a important simplifying assumption and a performance limitation. If there
  exists a leader, then prepare messages are superfluous. Intuitively, the leader implicitly holds
  a lock on all replicas because no other replica can initiate proposals. This allows proposals to
  be learned in just two message delays. However, the reliance on the leader to initiate all
  proposals is also a significant bottleneck at scale. The entire system moves at the rate of the
  leader. In fact, this is the fundamental limitation in implementations of Paxos like ZooKeeper
  ~\cite{zookeeper} and Chubby~\cite{chubby}. We will see in Egalitarian Paxos that we can remove
  the dependence on a leader to improve performance.

  \subsection{Generalized Paxos}
  Generalized Paxos~\cite{generalized-paxos} addresses the scalability of Paxos by exploiting
  commutativity. An operation $A$ \textbf{commutes} with $B$ if performing $A$ after $B$ has the
  same effect as performing $B$ after $A$. For example, addition is commutative but division is not;
  $4 + 3 = 3 + 4$ but $\frac{4}{3} \ne \frac{3}{4}$. In fact, most operations on a distributed
  database are commutative. Reads commute with each other and reads and writes to different keys
  commute.

  Generalized Paxos associates each proposal with a sequence of operations. We say that proposal $A$
  is \textbf{equivalent} to proposal $B$ if all non-commutative operations in $A$ and $B$ are in the
  same order. All equivalent proposals have the same effect. Generalized Paxos permits replicas to
  learn different proposals as long as they are equivalent.

  In Generalized Paxos, proposers do not forward their requests to leader. Instead, they immediately
  request that all replicas accept their proposed operation. A replica appends the operation to
  their currently accepted proposal and requests that all replicas learn it. A proposal is learned
  when a majority of replicas have requested that it or an equivalent proposal be learned.
  If no majority of replicas can agree on the ordering of non-commutative operations, it is the
  responsibility of the leader to select one and to run an instance of Paxos to convince the other
  replicas to accept its choice before resuming normal operation.

  Like Paxos, Generalized Paxos relies on the existence of a single, stable leader to mediate
  ordering disagreements between replicas and guarantees that all commutative operations will be
  learned in two message delays. Unlike Paxos, it does not require all proposals to originate from
  the leader. If most operations are commutative, the leader will rarely be required to arbitrate.
  However, the existence of a leader can still be a scalability bottleneck. We will see in
  Egalitarian Paxos that we can remove the dependence on a leader to improve the performance of the
  system.

  \subsection{Egalitarian Paxos}
  Egalitarian Paxos~\cite{epaxos} makes a subtle modification to Generalized Paxos to remove its
  dependence on a leader. Egalitarian Paxos associates with proposal with a directed acyclic graph
  of operations in which each edge corresponds to a dependency between two operations. The benefit
  of using a directed acyclic graph is that its various strongly connected components can be
  performed in parallel without impacting consistency. This has huge ramifications for performance,
  particularly in databases because reads and writes are relatively expensive operations.

  In Egalitarian Paxos, an operation depends on all accepted proposals for which it does not
  commute. The proposer builds a dependency graph for a proposal from any proposals that it has
  already accepted and requests that all replicas accept it. A replica supplements the dependency
  graph of the proposal with any proposals that it has accepted and requests that the result be
  learned. If no majority of replicas can agree on the dependency graph of a proposal, it is the
  responsibility of the proposer to select one and to run an instance of Paxos to convince the
  other replicas to accept its choice before resuming normal operation.

  Egalitarian Paxos implicitly assumes that operations are idempotent, because an operation may be
  in the dependency graph of multiple proposals. An operation $A$ is \textbf{idempotent} if repeated
  non-sequential applications of $A$ have the same effect as a single application of $A$. For
  example, multiplication by one is idempotent but by two is not; $4 * 1 = 4 * 1 * 1$ but
  $4 * 2 \ne 4 * 2 * 2$. This assumption will become important when we use Egalitarian Paxos to
  implement distributed transactions.

% Distributed Transactions.
\section{Distributed Transactions}
We now use Egalitarian Paxos to implement distributed transactions. We begin by concretely
defining the underlying system. We then describe the consensus protocol and conclude with a rough
proof of correctness.

  % Terminology.
  \subsection{Terminology}
  A \textbf{database} is a key-value store that supports two operations: \textbf{read} and
  \textbf{write}. Databases guarantee durability; if a key is written, then subsequent
  reads will always return the updated value. Like Caustic, we will associate each key with a
  \textbf{revision}, or versioned value, with the same conflict relationship as before. Reads and
  writes on the database return or update the revisions of a set of keys. Operations on a database
  are non-transactional and can fail arbitrarily.

  We define our abstract operation as a \textbf{transaction}. A transaction is composed of a set
  of dependent versions, called its \textbf{readset}, and a set of updates, called its
  \textbf{writeset}. We say that two transactions \textbf{conflict} if there exists a key in the
  readset or writeset of either transaction that is in the writeset of the other. A transaction may
  be \textbf{committed} on a database by applying the updates in its writeset if and only it
  depends on the latest version of every key in its readset. We say that a transaction is
  \textbf{committable} if it can be committed. In order to guarantee idempotency of commit, any key
  in the writeset of a transaction must also be present in its readset.

  A \textbf{cache} is a write-through database. Because dependencies are validated on commit, keys
  may be speculatively read from cache without sacrificing consistency. Beaker supports multi-level
  cache hierarchies over the underlying database. Revisions may be \textbf{fetched} or
  \textbf{updated} in cache. Cache coherency is maintained by updating the cache whenever
  transactions are committed on the underlying database.

  An \textbf{archive} is a transactional database. Archives use an \textbf{executor} to linearize
  conflicting transactions on the underlying database. An executor schedules transactions in
  \textbf{epochs}. Transactions are scheduled in the earliest epoch such that they do not conflict
  with any transaction in a later epoch. This guarantees consistency and isolation; transactions
  see the effect of any previously committed conflicting transaction and conflicting transactions
  are never committed simultaneously.

  We define our replica as a \textbf{beaker}. Beakers use a variation of Egalitarian Paxos to
  coordinate distributed transactions with several desirable properties. First, non-conflicting
  transactions may be simultaneously committed. Second, databases with stale revisions are
  automatically repaired. Third, transactions may be committed as long as at least a majority of
  beakers are operational.

  % Protocol.
  \subsection{Protocol}
  Beaker makes two key modifications to Egalitarian Paxos. First, it associates each proposal with
  a set of non-conflicting transactions, called its \textbf{commits}, and a set of revisions,
  called its \textbf{repairs}. We say that a proposal $A$ \textbf{conflicts} with a proposal $B$ if
  they have any conflicting commits. We say that a proposal $A$ is \textbf{equivalent} to proposal
  $B$ if their commits are the same. A proposal $A$ may be \textbf{merged} with a proposal $B$ by
  taking the maximum of their ballots, combining their commits choosing the transaction in the
  newer proposal in the case of conflicts, and combining their repairs choosing the highest
  revision in the case of duplicates. Second, it introduces an intermediate \textbf{get} phase to
  the protocol. We will see that this additional phase allows the system to repair databases with
  stale revisions.

  In Beaker, the proposer prepares a proposal by sending it to all beakers. If a beaker has not
  made a promise to a newer proposal, it responds with a promise not to accept any proposal that
  conflicts with the proposal it returns. If it has already accepted any conflicting proposals, it
  merges them together and returns the result. Otherwise, it returns the prepared proposal with a
  zero ballot. If the proposer does not receive a majority of promises, it retries with a higher
  ballot. Otherwise, it merges the returned promises together into a single proposal. If the
  merged proposal is not equivalent to its prepared proposal, the proposer retries with a higher
  ballot. Otherwise, the proposer gets the revisions of all keys that are read by the proposal
  from a majority of beakers. The proposer discards all transactions that cannot be committed
  given the latest returned revisions and sets the repairs of the proposal to the latest revisions
  of keys that are read - but not written - by the proposal for which any two beakers have
  different revisions. The proposer then requests all beakers to accept the proposal. A
  beaker accepts a proposal if it has not promised not to. If a beaker accepts a proposal, it
  requests that all other beakers learn the proposal. A beaker learns a proposal when a majority
  of beakers have requested that it be learned. When a beaker learns a proposal, it commits its
  transactions and repairs on its archive.

  % Correctness.
  \subsection{Correctness}
  We will make the assumption of \textbf{connectivity}; all learn messages are always delivered.
  This ensures that the proposer will always learn that its proposal was learned. This is a
  necessary requirement because it allows clients to know with certainty whether or not their
  transaction was committed. In practice, this assumption can be weakened by adding an additional
  decide phase to the protocol. Whenever a proposal is learned by a beaker, it informs all other
  beakers that the proposal was decided. As long as at least one beaker receives a majority of
  learn messages and at least one decide message is delivered to each beaker, any proposer is
  guaranteed to learn that its proposal was learned. A proposal is learned by a beaker when it
  receives either a majority of learn requests or a decide message. The decide phase introduces
  additional partition tolerance and comes at no additional cost if the network is perfectly
  reliable. It does, however, increase the number of messages sent between beakers which could
  saturate the network. The assumption of connectivity is reasonable if beakers are co-located in
  the same data center, but is unreasonable if they are geographically separated.

  We will also make the assumption of \textbf{reliability}; at least a majority of replicas will
  atomically apply each learned proposal. This ensures that any majority of replicas will
  contain at least one that has the latest revision of each key-value pair. Our assumption of
  reliability is substantially weaker than the one implicitly made by the previously presented Paxos
  algorithms; they assume that a majority of replicas atomically apply \emph{every} learned
  proposal. We will see that consistency is guaranteed even if no replica has applied every learned
  proposal.

  We will also make use of the fact of \textbf{quorum intersection}; any two majorities of beakers
  must have at least one in common. Verification of this property is trivial and is left to the
  reader. \\

  \begin{theorem}[Liveness]
    Any proposal $A$ that is accepted by a majority will eventually be learned by everyone.
  \end{theorem}
  \begin{proof}
    By quorum intersection, at least one promise will contain $A$ until $A$ has been learned by
    sufficiently many beakers that it no longer has been accepted by a majority. But if $A$ was
    learned by any beaker, at least a majority of beakers must have requested that it be learned.
    By connectivity, if a majority of beakers requested that $A$ be learned then all beakers will
    learn $A$. Therefore, $A$ will eventually be learned by everyone.
  \end{proof}

  Note that we guarantee only that a proposal will eventually be learned once it has been accepted
  by a majority, but we do not guarantee that a proposal will ever accepted by a majority. It is
  possible for proposers to continuously prepare higher ballot proposals before any can be
  accepted by a majority. In practice, by retrying with exponentially jittered backoffs the
  likelihood of such an event asymptotically approaches zero. \\

  \begin{theorem}[Serializability]
    If proposal $A$ is accepted by a majority, then any conflicting proposal $B$ that is
    subsequently prepared will always be learned after $A$.
  \end{theorem}
  \begin{proof}
    Because $A$ was accepted by a majority before $B$, the majority that accepted $A$ before $B$
    will request that $A$ be learned before $B$. Because messages are always delivered in order
    and by connectivity, $B$ will always be learned after $A$.
  \end{proof}

  \begin{theorem}[Commutativity of Repairs]
    Let $R$ denote the repairs for a proposal $A$ that has been accepted by a majority. Any proposal
    $B$ that is accepted by majority and conflicts with $A + R$ but not $A$ commutes with $A + R$.
  \end{theorem}
  \begin{proof}
    Because $B$ conflicts with $A + R$ but not $A$, $B$ must read a key $k$ that is read by $A$.
    By reliability, $B$ must read the latest version of $k$. Suppose that $B$ is committed first.
    Because $B$ reads and does not write $k$, $A + R$ can still be committed. Suppose that
    $A + R$ is committed first. Because $A + R$ writes the latest version of $k$ and $B$ reads the
    latest version, $B$ can still be committed. Therefore, $B$ commutes with $A + R$.
  \end{proof}

  \begin{theorem}[Consistency]
    If a proposal $A$ is accepted by a majority, its transactions are committable.
  \end{theorem}
  \begin{proof}
    Suppose there exists a transaction that cannot be committed. Then, the transaction must read a
    key for which there exists a newer version. This implies that there exists a proposal $B$ that
    was accepted by a majority after, but learned before $A$ that changes a key $k$ that is read
    by $A$. By serializability, $B$ cannot conflict with $A$. Therefore, $B$ must repair $k$. By
    commutativity of repairs, $A$ may still be committed.
  \end{proof}

  % Reconfiguration.
  \subsection{Reconfiguration}
  In practical systems, beakers may join or leave the cluster arbitrarily as the cluster grows or
  shrinks in size. In this section, we describe how beakers are bootstrapped when they join an
  existing cluster. We say that a beaker is \textbf{fresh} when it initially joins a cluster. In
  order to guarantee correctness, fresh beakers must be immediately populated with the latest
  revision of every key-value pair. Otherwise, if $N + 1$ fresh beakers join a cluster of size $N$
  it is possible for a majority of beakers to consist entirely of fresh beakers. This violates the
  assumption of reliability.

  A naive solution might be for the fresh beaker to propose a read-only transaction that depends on
  the initial revision of every key-value pair in the database and conflicts with every other
  proposal. Then, the fresh beaker would automatically repair itself in the process of committing
  this transaction. However, this is infeasible in practical systems because databases may contain
  arbitrarily many key-value pairs. This approach would inevitably saturate the network.
  Furthermore, it prevents any proposals from being accepted in the interim.

  We can improve this solution by decoupling bootstrapping and consensus. A fresh beaker joins the
  cluster as a partial member, called a \textbf{learner}, that learns proposals, but is not involved
  in consensus. The fresh beaker \textbf{scans} a majority of full members, called
  \textbf{acceptors}, and repairs its own archive with the latest returned values. By assumption of
  reliability, it is guaranteed to have the latest value of every key. It then joins the cluster as
  an acceptor. This approach consumes less bandwidth and permits concurrent proposals.

  Each beaker maintains its own view of the cluster configuration which it sends alongside any
  proposal that it proposes. If the view is outdated, beakers inform the proposer of the new
  configuration. If a beaker has already accepted a proposal from an outdated view, that proposal
  must be completed before the cluster can move to the new configuration. This ensures that
  reconfiguration is consistent across the cluster.

% Evaluation.
\section{Evaluation}

  % YCSB.
  \subsection{YCSB}
  The \href{https://github.com/brianfrankcooper/YCSB}{Yahoo Cloud Serving Benchmark} (YCSB) is a
  common benchmark used to evaluate databases under various workloads. Each workload is composed of
  a different ratio of reads, inserts, and updates to benchmark performance under various kinds of
  load. Workload A is composed 50\% reads and 50\% updates, B of 95\% reads and 5\% updates, C of
  100\% reads, D of 95\% reads and 5\% inserts, and F of 50\% reads and 50\% read-modify-writes.

  \begin{figure}
    \caption{YCSB Throughput}
    \label{ycsb-throughput}
    \centering
    \begin{tikzpicture}
      \begin{axis}[
        ybar,
        xlabel={Instances},
        ylabel={Throughput (ops/sec)},
        xtick=data,
        bar width=8,
        enlarge x limits={abs=0.6},
        legend cell align={left},
        legend style={at={(1.03,1)}, anchor=north west, draw=black, fill=white, align=left},
      ]
        \addplot
        table[y index=1] {data/beaker-throughput.dat};
        \addlegendentry{Workload A};
        \addplot
        table[y index=2] {data/beaker-throughput.dat};
        \addlegendentry{Workload B};
        \addplot
        table[y index=3] {data/beaker-throughput.dat};
        \addlegendentry{Workload C};
        \addplot
        table[y index=4] {data/beaker-throughput.dat};
        \addlegendentry{Workload D};
        \addplot
        table[y index=5] {data/beaker-throughput.dat};
        \addlegendentry{Workload F};
      \end{axis}
    \end{tikzpicture}
  \end{figure}

  \begin{figure}[h]
    \caption{YCSB Read Latency}
    \label{ycsb-latency-read}
    \centering
    \begin{tikzpicture}
      \begin{axis}[
        ybar,
        xlabel={Instances},
        ylabel={95\% Read Latency ($\mu$s)},
        xtick=data,
        ymin=0,
        bar width=8,
        enlarge x limits={abs=0.6},
        legend cell align={left},
        legend style={at={(1.03,1)}, anchor=north west, draw=black, fill=white, align=left},
      ]
        \addplot
        table[y index=1] {data/beaker-latency-read.dat};
        \addlegendentry{Workload A};
        \addplot
        table[y index=2] {data/beaker-latency-read.dat};
        \addlegendentry{Workload B};
        \addplot
        table[y index=3] {data/beaker-latency-read.dat};
        \addlegendentry{Workload C};
        \addplot
        table[y index=4] {data/beaker-latency-read.dat};
        \addlegendentry{Workload D};
        \addplot
        table[y index=5] {data/beaker-latency-read.dat};
        \addlegendentry{Workload F};
      \end{axis}
    \end{tikzpicture}
  \end{figure}

  \begin{figure}[h]
    \caption{YCSB Write Latency}
    \label{ycsb-latency-write}
    \centering
    \begin{tikzpicture}
      \begin{axis}[
        ybar,
        xlabel={Instances},
        ylabel={95\% Write Latency ($\mu$s)},
        xtick=data,
        ymin=0,
        bar width=8,
        enlarge x limits={abs=0.6},
        legend cell align={left},
        legend style={at={(1.03,1)}, anchor=north west, draw=black, fill=white, align=left},
      ]
        \addplot
        table[y index=1] {data/beaker-latency-write.dat};
        \addlegendentry{Workload A};
        \addplot
        table[y index=2] {data/beaker-latency-write.dat};
        \addlegendentry{Workload B};
        \addplot
        table[y index=3] {data/beaker-latency-write.dat};
        \addlegendentry{Workload C};
        \addplot
        table[y index=4] {data/beaker-latency-write.dat};
        \addlegendentry{Workload D};
        \addplot
        table[y index=5] {data/beaker-latency-write.dat};
        \addlegendentry{Workload F};
      \end{axis}
    \end{tikzpicture}
  \end{figure}

  In Figure~\ref{ycsb-throughput}, we examine how throughput changes as the number of instances in
  the cluster increases. We see that throughput scales well in read-dominant workloads and falls
  considerably in write-dominant workloads. In Figure~\ref{ycsb-latency-read}, we examine the 95\%
  read latency as the number of instances in the cluster increases. We see that read latency remains
  relatively constant across all workloads, even as the number of instances increases. In
  Figure~\ref{ycsb-latency-write}, we examine the 95\% read latency as the number of instances in
  the cluster increases. We see that write latency increases as the number of instances increases,
  because more instances take longer to reach consensus.

  % Contention.
  \subsection{Contention}
  In this section, we examine the performance of Beaker under contention both from a theoretical
  and a practical perspective. We begin with a mathematical treatment of consensus and conclude with
  an empirical verification of our analysis.

  We may model the number of retries required to successfully commit a transactions as a negative
  binomial distribution. A negative binomial distribution $NB(p, r)$ measures the number of
  failed events each occurring with probability $p$ before $r$ successes are encountered. In
  this case, $p$ represents the \textbf{contention probability}, or the likelihood that any two
  concurrent transactions conflict. Therefore, the distribution of total attempts required to
  successfully commit any transaction as $A \sim 1 + NB(p, 1)$. We may then use known results about
  the negative binomial distribution to make predictions about the how the system will behave under
  contention.

  We must first make concrete our definition of contention probability. Consider two sets $X$ and
  $Y$ of size $l$ selected uniformly at random from the set [0, n). The contention probability is
  the likelihood that $X$ and $Y$ are not disjoint. Intuitively, the probability that $X$ and $Y$
  are disjoint is equal to the likelihood that $Y$ is drawn from the set $[0, n) \setminus X$.
  Therefore, the probability that $X$ and $Y$ are not disjoint is equal to
  $1 - \frac{C(n - 1, l)}{C(n, l)}$ where $C(a, b)$ is the number of ways to choose $b$ items from a
  set of size $a$. We may extend this notion of uniformly random sets to uniformly random
  transactions. Suppose two transactions $T_1$ and $T_2$ each write $l$ keys chosen uniformly at
  random from a key space of size $n$. The probability that they will conflict is exactly equal to
  the probability that $X$ and $Y$ are not disjoint.

  To verify that this theoretical definition of contention probability matches empirical results I
  conducted a Monte Carlo simulation in which two threads repeatedly generate and commit $M$
  transactions that increment a uniformly random set of $l$ keys and monitor the total number of
  failures. If the hypothesis about the negative binomial distribution is correct, transactions
  should conflict on average $p * M$ times. I found this to be empirically validated, but the reader
  is encouraged to verify this claim for themselves.

  We assume for the sake of simplicity that there exist exactly two proposers that are
  simultaneously committing transactions. In practice, most systems will have many more active
  proposers. Unfortunately, I lack the mathematical ability to analytically derive contention
  probability in the general case. However, the contention probability may be numerically
  approximated by changing the number of threads in the Monte Carlo simulation.

  We may use our empirically verified model of contention to make predictions about the performance
  degradation of the system under varying degrees of contention. Given a key-space of size $n$ and
  an average transaction of size $l$, we would expect each transaction to require
  $\mathbb{E}[A] = 1 + \frac{p}{1 - p}$ attempts and for the average throughput of the system to
  decrease by the same amount.

% Conclusion.
\section{Conclusion}
We have presented the design and implementation of Beaker and shown that it is possible to build
robust programming language from the minimal interface it provides.

\end{document}
